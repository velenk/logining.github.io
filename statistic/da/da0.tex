\documentclass[UTF8]{ctexart}
\usepackage{verbatim}
\usepackage{amsmath}
\usepackage{amsfonts}
\usepackage{enumerate}
\begin{document}
\pagestyle{headings}
\title{Data Analysis}
\author{Velen~Kong
%\thanks{Professer Zhao}
}
\maketitle
\begin{abstract}

基于Python的数据分析入门笔记,希望能自学掌握基本的数据分析能力,毕竟是统计学的基础之一。

内容安排主要参考

(美) Wes McKinney 著. Python for Data Analysis:2nd Edition [M] USA:O'Reilly Media 2017

该书作者同时也是pandas库的作者,同时借用其在GitHub上的数据资料。

\end{abstract}

\section{第三方库与Python入门}

\subsection{IPython \& Jupyter}

用到的库:ipython,jupyter。

jupyter确实好用,主要扩展了$Tab,?,*$的功能。

还有一些快捷键仅限IPython使用。

另外就是一些常用的Magic Command,比如$\%run, \%load, \\
\%debug, \%xmode~Plain, \%matplotlib~inline$

\subsection{Python基础}

\#注释。

所有变量都是对象(object)。

基本的函数调用,利用函数批量处理。

变量赋值类似引用(reference)。

动态类型,利用type()和isinstance()进行类型检查。后者可以传入tuple代替逻辑或操作。

print配合format输出。

Python中的object拥有各自的属性(attributes)和方法(methods),可通过getattr, hasattr, setattr操作。其中getattr可以直接使用返回对象,setattr不改变原class。





\section{Numpy库}

\end{document}








