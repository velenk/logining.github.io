\documentclass[UTF8]{ctexart}
\usepackage{verbatim}
\usepackage{amsmath}
\usepackage{amsfonts}
\usepackage{enumerate}
\newtheorem{law}{定理}[section]
\newtheorem{lemma}{引理}[section]
\begin{document}
\pagestyle{headings}
\title{Mathematical Statistics}
\author{Velen~Kong
%\thanks{Professer Zhao}
}
\maketitle
\begin{abstract}
出于下学期的课程以及数学建模的需要,只好暑假自学数理统计,自学首选课本为

韦来生.著. 数理统计(第二版) [M] 北京:科学出版社 2015

由于本人水平限制,大概率不会涉及以下参考资料

陈希孺.著. 高等数理统计学 [M] 合肥:中国科学技术大学出版社 2009

(美) Hogg R.V. \& Craig A.T.著. Introduction to Mathematical Statistics:Fifth Edition 数理统计学导论:第5版(影印版) [M] 北京:高等教育出版社 2004

(美) Casella G. \& Berger R.L.著. Statistical Inference 统计推断(原书第二版) [M] 张忠占 \& 傅莺莺.译. 北京:机械工业出版社 2009\\
\end{abstract}

\section{绪论及预备知识}

几个基本概念,总体(population),个体(individual),样本(sample),抽样(sampling),随机变量(random variable),观察值(observation)。总体又分为有限总体(finite population)和无限总体(infinite population)。

然后又有样本空间(sample space),简单随机样本,联合分布函数和联合密度函数
$$F(x_{1},\ldots ,x_{n})=\prod_{i=1}^{n}F(x_{i})$$

$$f(x_{1},\ldots ,x_{n})=\prod_{i=1}^{n}f(x_{i})$$

对于多维样本同样有
$$F(X,Y)=\prod_{i=1}^{n}F(x_{i},y_{i})$$

$$f(X,Y)=\prod_{i=1}^{n}f(x_{i},y_{i})$$

确定统计模型(statistical model)后,从总体中抽取一定的样本来推断总体模型被称为统计推断(statistical inference)。需要估计的是参数向量(parameter vector),参数的取值范围就是参数空间(parameter space)。除此以外还有未知样本分布的非参数统计推断。

而由于参数的不确定性,统计模型是一个样本分布族(distribution family of the sample)。

统计量(statistic)则是样本算出的值,例如样本均值(mean)和方差(variance)。下面介绍均值和方差的推广——样本矩(sample moments)。

设$X_{1},\ldots ,X_{n}$是总体中抽取的样本,则称
$$a_{n,k}=\frac{1}{n}\sum_{i=1}^{n}X_{i}^{k},\qquad k=1,2,\ldots $$

为样本$k$阶原点矩。
$$m_{n,k}=\frac{1}{n}\sum_{i=1}^{n}(X_{i}-\overline{X})^{k} $$

为样本$k$阶中心矩。

对于二维总体而言,称
$$S_{XY}=\frac{1}{n}\sum_{i=1}^{n}(X_{i}-\overline{X})(Y_{i}-\overline{X}) $$

为$X$和$Y$的样本协方差(sample covariance)。

然后我们介绍次序统计量(order statistics)。将$X_{1},\ldots ,X_{n}$按递增次序排列为$X_{(1)},\ldots ,X_{(n)}$就是样本的次序统计量。

利用次序统计量可以定义样本中位数(sample median),样本极值(extremum of sample),样本$p$分位数(sample p-fractile),样本极差(sample range)。其中$p$分位数定义为$X_{(m)},m=[(n+1)p]$。

样本变异系数(sample coefficient of variation)则是对于总体变异系数(population coefficient of variation)的估计。总体变异系数衡量总体分布的散布程度,定义为
$$\nu =\frac{SD(X)}{E(X)} $$

而样本变异系数则为
$$\nu = \frac{S_{n}}{\overline{X}} $$

样本偏度(sample skewness)反映了总体偏度的信息,定义是
$$\beta_{1}=\frac{\mu_{3}}{\mu_{2}^{3/2}} $$

其中正态分布的偏度为$0$。

而样本偏度则定义为
$$\hat{\beta_{1} }=\frac{m_{n,3}}{m_{n,2}^{3/2}} $$

样本峰度(sample kurtosis)则是用来反映总体峰度的信息,总体峰度表示密度函数在最大值附近的集中程度,正态分布的峰度为$0$。总体峰度定义为
$$\beta_{2}=\frac{\mu_{4}}{\mu_{2}^{2}}-3 $$

而样本峰度则定义为
$$\hat{\beta_{2}}=\frac{\mu_{4}}{\mu_{2}^{2}}-3 $$

接下来介绍经验分布函数,对于一组次序统计量,我们定义
$$F_{n}(x)=\left\{ \begin{array}{ll}
0, & x\leq X_{(1)},\\
\frac{k}{x}, & X_{(k)}<x\leq X_{(k+1)}\\
1, & X_{(n)}<x
\end{array} \right. $$

为经验分布函数(empirical distibution function)。易见经验分布函数是单调不减左连续的,它可以看作总体分布函数的一个估计量。

若记示性函数为
$$I_{A}(x)=\left\{ \begin{array}{ll}
1, & x\in A \\
0, & others
\end{array} \right. $$

那么$F_{n}(x)$可以表示为
$$F_{n}(x)=\frac{1}{n}\sum_{i=1}^{n}I_{(-\infty ,x)}(X_{i}) $$

所以有
$$nF_{n}(x)=\sum_{i=1}^{n}Y_{i}~b(n,F(x)) $$





\section{抽样分布}

\section{点估计}

\end{document}








